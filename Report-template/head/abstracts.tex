% English abstract
\cleardoublepage
\chapter*{Tóm tắt}
%\markboth{Abstract}{Abstract}
\addcontentsline{toc}{chapter}{Tóm tắt} % adds an entry to the table of contents
\vspace{1.0cm}
Gần đây, khái niệm Cách mạng Công nghiệp 4.0 được nhắc đến nhiều trên truyền thông và mạng xã hội. Cùng với đó là những hứa hẹn về cuộc đổi đời của các doanh nghiệp tại Việt Nam nếu đón được làn sóng này. Cách mạng Công nghiệp 4.0 sẽ diễn ra trên 3 lĩnh vực chính gồm Công nghệ sinh học, Kỹ thuật số và Vật lý. Những yếu tố cốt lõi của Kỹ thuật số sẽ là: Trí tuệ nhân tạo (AI), Vạn vật kết nối - Internet of Things (IoT) và dữ liệu lớn (Big Data)\cite{cmcn}. Trong đó IoT sẽ đảm nhiệm việc kết nối, thu nhập thông tin từ các thiết bị và vật dụng tạo nên Big Data, còn AI sẽ đảm nhiệm việc xử lý chúng.\\

Hiện nay các doanh nghiệp tham gia phát triển các hệ thống IoT sử dụng rất nhiều chuẩn kết nối khác nhau, có những chuẩn quen thuộc với đa số mọi người như WiFi, Ethernet, Bluetooth; cũng có những chuẩn thuộc dạng \lq\lq{}chuyên ngành\rq\rq{} như ZigBee, LoRa, Xbee, RFID. Mỗi chuẩn đều có những ưu nhược điểm riêng và phù hợp với các ứng dụng cụ thể.\\

Trong đề tài này, nhóm sẽ thảo luận về giao thức Bluetooth Mesh, một giao thức mạng được phát triển dựa trên chuẩn giao tiếp Bluetooth. Bluetooh Mesh là một chuẩn ra đời sau, khá mới so với những chuẩn khác nên khả năng chiếm giữ thị trường còn hạn chế. Tuy nhiên với những ưu điểm đầy hứa hẹn của mình, Bluetooh Mesh sẽ nhanh chóng thay đổi bộ mặt của lĩnh vực IoT.\\

Ngoài việc tổng hợp một số nghiên cứu của nhóm về nguyên lý hoạt động của Bluetooh Mesh để làm tài liệu tham khảo cho các khóa sau, nhóm còn tiến hành thử nghiệm hiện thực một ứng dụng đơn giản để chứng minh tiềm năng của Bluetooh Mesh. Ứng dụng bao gồm việc thu thập dữ liệu và cập nhật lên cloud thông qua mạng GPRS.


\vskip0.5cm
%put your text here


%\endgroup			
%\vfill
