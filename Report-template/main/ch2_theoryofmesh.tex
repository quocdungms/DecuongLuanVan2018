\chapter{Nội dung thực hiện} \label{chap:theory}
    \section{Các công việc liên quan đến giám sát hệ
    	thống và quản trị mạng}
    	\begin{itemize}
    		\item Xác định, khắc phục sự cố, giải quyết và ghi lại các vấn đề kết nối và hiệu suất mạng.
    		\item Giám sát hiệu suất mạng và tối ưu hóa mạng để có tốc độ và tính sẵn sàng tối ưu.
    		\item Cài đặt, cấu hình và duy trì phần cứng mạng.
    		\item Triển khai, cấu hình và nâng cấp phần mềm mạng.
    		\item Triển khai và duy trì các hệ thống sao lưu và khôi phục khẩn cấp cho các máy chủ mạng quan trọng.
    		\item Giám sát tình hình hoạt động, tài nguyên của các thành phần trong hệ thống.
    		\item Quản trị quyền truy cập của người dùng vào những file nhạy cảm.
    		\item Quản lí user/group trên hệ thống.
    		\item Quản lí các phần mềm trên hệ thống.
    		\item Quản lý các công cụ bảo mật mạng,...
    	\end{itemize}

   	\section{Các công cụ được sử dụng để giám sát hệ thống và quản trị mạng}
   		\subsection{Giám sát hệ thống}
   		\textbf {Webmin}
   		
   		Chức năng nổi bật :
	   		\begin{itemize}
	   			\item Quản lí user/group trên hệ thống.
	   			\item Quản lí phần mềm trên hệ thống.
	   			\item Cấu hình thời gian cho hệ thống.
				\item Quản lí File.
				\item Thực thi lệnh linux.
	   		\end{itemize}
   	
   		Ưu nhược điểm :
   			\begin{itemize}
   				\item Miễn phí và có đầy đủ tính năng quan trọng.
   				\item Tương thích với WordPress.
   				\item Không thể tạo ra các host con để chia sẻ dùng chung.
   			\end{itemize}
   		\textbf {Virtualmin}
   		
   		Chức năng nổi bật :
   		\begin{itemize}
   			\item Quản lí giới hạn tài nguyên sử dụng.
   			\item Hỗ trợ backup và tự backup.
   			\item Giao diện thân thiện.
   			\item Quản lí user và tạo host.
   			\item Xem và quản lí các thiết lập về máy chủ, mạng.
   		\end{itemize}
   		
   		Ưu nhược điểm :
   		\begin{itemize}
   			\item Linh hoạt hơn, có thể định cấu hình, giao diện người dùng.
   			\item Có giao diện đẹp trên cả tablet và mobile.
   			\item Không thể tạo ra các host con để chia sẻ dùng chung.
   		\end{itemize}
   	
   		Hệ thống hỗ trợ :
   		\begin{itemize}
   			\item Ubuntu, CentOS/RHEL6 \& 7, Debian.
   		\end{itemize}