\chapter{Giới thiệu đề tài}
\pagestyle{fancy}
Trong nền công nghệ phát triển như hiện nay, khi mà công nghệ chế tác các thiết
bị phần cứng đang đạt đến mức bão hòa ( ít đột biến ) thì bài toán về việc tối
ưu hóa ( tăng tính tương thích giữa thiết bị phần cứng và phần mềm, giảm tải
nguyên sử dụng, tăng hiệu suất hoạt động của hệ thống...) luôn được đặc biệt quan tâm. Để có hướng giải quyết bài
toán đó, trước hết chúng ta cần phải nắm rõ về hệ thống đó, biết được hệ thống
đang hoạt động ra sao, tốn bao nhiêu tài nguyên cho một tác vụ...

Từ đó, ta sẽ dễ dàng đưa ra được hướng giải quyết cho bài toán tối ưu. Trên thị
trường đã có các công cụ quản lí tài nguyên hệ thống như : Webmin, VirtualMin,
VestaCP, ISPconfig... hay công cụ quản trị mạng như : Nagios, Icinga, Zenoss,
Cisco Network Assistant... Tuy nhiên, gần như rất ít công cụ tích hợp quản trị tài nguyên và quản trị mạng lại với nhau, nếu có thì giá thành của nó vẫn còn khá cao. Bởi vậy việc cho ra một công cụ tối ưu giúp người giám sát hệ thống có được cái nhìn trực quan nhất về mọi mặt của hệ thống là điều hết sức cần thiết.

    \section{Nhiệm vụ cần đạt}
    \begin{itemize}
        \item Tìm hiểu các công việc liên quan đến việc giám sát, quản lý hệ thống và quản trị mạng.
        \item Tìm hiểu về các công cụ được sử dụng để giám sát hệ thống và quản trị mạng.
        \item Thiết kế mô hình ứng dụng, các tính năng và phương pháp tích hợp.
    \end{itemize}
   